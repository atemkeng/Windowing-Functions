% Appendix A

\section{The Fourier Transform of a top-hat windowing function} %Main appendix title
\label{AppendixA} % For referencing this appendix elsewhere, use \ref{AppendixA}
%\lhead{Appendix A. \emph{The sampled function}} % This is for the header on each page - perhaps a shortened title
\begin{alignat}{2}
\Pi_{lm}(u_{t\nu},v_{t\nu}) = \left\{
\begin{array}{rl}
1 & \mbox{for $\mathit{ t \times \nu \in  [t_s, t_e]\times [\nu_s, \nu_e]}$}, \\
0 & \mbox{for $\mathit{ t \times \nu  \notin [t_s, t_e]\times [\nu_s, \nu_e]}$ }
\end{array}\right.\label{eq:box}
\end{alignat}
where $t_s$, $t_e$, $\nu_s$ and $\nu_e$ are the starting times, ending time, starting frequency and ending frequency
 respectively. The Fourier transform, $\check{\Pi}_{lm}(u_{t\nu},v_{t\nu})$ of Eq.\ref{eq:box} is given by
\begin{alignat*}{2}
\check{\Pi}_{lm}(u_{t\nu},v_{t\nu})&=sinc\frac{-2\pi t\Delta \nu}{2}sinc\frac{-2\pi\nu\Delta t}{2}
\end{alignat*}
Let supose that 


